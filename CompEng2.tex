%%%%%%%%%%%%%%%%%%%%%%%%%
% Dokumentinformationen %
%%%%%%%%%%%%%%%%%%%%%%%%%
\newcommand{\titleinfo}{CompEng2 Zusammenfassung}
\newcommand{\authorname}{\href{mailto:lmazzole@hsr.ch}{L. Mazzoleni}}
\newcommand{\authoremail}{\href{mailto:lmazzole@hsr.ch}{lmazzole@hsr.ch} }
\newcommand{\versioninfo}{}
\RequirePackage{luatex85}
\def\pgfsysdriver{pgfsys-pdftex.def}

%\newcommand{\authorinfo}{Stefan Reinli \texttt{stefan.reinlil@hsr.ch}\\Luca Mazzoleni \texttt{luca.mazzoleni@hsr.ch}}
%NEWCOMMANDS überarbeiten
%wieso mit new Command? Author Email Info besser aufteilen

%%%%%%%%%%%%%%%%%%%%%%%%%%%%%%%%%%%%%%%%%%%%%
% Standard projektübergreifender Header für 
% - Makros 
% - Farben
% - Mathematische Operatoren
%
% dORT NUR ERGÄNZEN, NICHTS LÖSCHEN
%%%%%%%%%%%%%%%%%%%%%%%%%%%%%%%%%%%%%%%%%%%%%
%BuG-Fix
%Package pdf Error: Driver file ................ not found
%If you have a luatex driver fail uncomment these lines
\RequirePackage{luatex85}
\def\pgfsysdriver{pgfsys-pdftex.def}

% Genereller Header
\documentclass[11pt,twoside,a4paper,fleqn]{article}
% Dateiencoding
\usepackage[utf8]{inputenc}
\usepackage[T1]{fontenc}	%ä,ü...
% Seitenränder
\usepackage[left=1cm,right=1cm,top=0.5cm,bottom=0.5cm,includeheadfoot]{geometry}
% Sprachpaket 
\usepackage[english, ngerman]{babel} % Silbentrennung und Rechtschreibung Englisch und Deutsch

%%%%%%%%%%%%%%%%%%%%%%%
%% Wichtige Packages %%
%%%%%%%%%%%%%%%%%%%%%%%
\usepackage{amsmath}                % Allgemeine Matheumgebungen									
\usepackage{amssymb}                % Fonts: msam,msbm, eufm & Mathesymbole, Mengen (lädt automatisch amsfonts)									
\usepackage{array}                  % \newcolumntype, \firsthline, ,\lasthline, m{width}, b{width}									
\usepackage{caption}                % Bildunterschriften									
\usepackage{enumitem}               % basic environments: enumerate, itemize, description									
\usepackage{fancybox}               % \fbox: \shad­ow­box, \dou­ble­box, \oval­box, \Oval­box									
\usepackage{fancyhdr}               % Seiten schöner gestalten, insbesondere Kopf- und Fußzeile									
\usepackage{floatflt}               % Textumflossene Abbildungen \begin{floatingfigure}[r]{Breite} : r rechts, l links, p links auf geraden Seiten und rechts auf ungeraden Seiten								
\usepackage{graphicx}               % \includegraphics[keyvals]{imagefile}, [draft]graphicx zeigt nur Namen und Rahmen an, [final] hebt diese option auf => Bild wird angezeigt    									
\usepackage{hyperref}               % Erstellt Verweise innerhalb und nach außerhalb eines PDF Dokumentes.									
\usepackage{lastpage}               % Bspw. : Page 1 of 3 => \thepage\ of \pageref{LastPage}									
\usepackage{listings}               % Erlaubt es Programmcode in der gewünschten Sprache zu hinterlegen (C++, Matlab,..). Definition der Sprache mit \lstset{language=name}..									
\usepackage{longtable}              % Longtable erlaubt es Tabellen zu erstellen die bei der nächsten Seite weiterlaufen. (Bricht automatisch um)									
\usepackage{mathabx}                % Mathesymbole									
\usepackage{mathrsfs}               % \mathscr (Benötigt für Fourierreihen-Symbol)									
%\usepackage{mathtools}              % Extension package to amsmath									
\usepackage{multicol}               % multicols-Umgebung \begin{multicols}{3} erzeugt Abschnitt mit 3 Spalten									
\usepackage{multirow}               % Tabelle: ermöglicht es Felder mehrerer Zeilen in einem zusammenzufassen									
\usepackage{pdflscape}              % adds PDF support to the environment 'landscape'									
\usepackage{pxfonts}                % Symbole, griechisches Alphabet, Integrale...									
\usepackage{rotating}               % sideways, turn{degree}, rotate{degree}, sidewaysfigure, sidewaystable Umgebung									
\usepackage{subcaption}             % Bildunterschriften für Subfigures									
\usepackage{tabularx}               % tabularx-Umgebung: Hat feste Gesamtbreite, \begin{tabularx}{\textwidth}{c c c c c} X: Spalte mit variabler Breite, l, c, r, p{breite}, m{breite}									
\usepackage{textcomp}               % text symbols: baht, bullet, copyright, musical-note, onequarter, section, yen									
\usepackage{tikz}                   % Tikz Umgebung zur Grafikerzeugung									
\usepackage{titlesec}               % Überschriften zu Textabstände
\usepackage{trfsigns}               % Transformationszeichen \laplace, \Laplace..									
\usepackage{trsym}                  % Weitere Laplace Zeichen erlaubt auch vertikale Transformationszeichen									
\usepackage{verbatim}               % verbatim, verbatim*, comment Umgebung									
\usepackage{wrapfig}                % Textumflossene Bilder und Tabellen, \begin{wrapfigure}[Zeilen]{Position}[Ueberhang]{Breite}									
\usepackage{xcolor}                 % \pagecolor{color}, \textcolor{color}{text}, \colorbox{color}{text}, \fcolorbox{border-color}{fill-color}{text}									
\usepackage{titlesec}
% Zum Bilder einfach in Tabellen einfügen (valign=t)
\usepackage[export]{adjustbox}

%%%%%%%%%%%%%%%%%%%%
% Generelle Makros %
%%%%%%%%%%%%%%%%%%%%
\newcommand{\skript}[1]{$_{\textcolor{red}{\mbox{\small{Skript S.#1}}}}$}
\newcommand{\verweis}[2]{\small{(siehe auch \ref{#1}, #2 (S. \pageref{#1}))}}
\newcommand{\verweiskurz}[1]{(\small{siehe \ref{#1}\normalsize)}}
\newcommand{\subsubadd}[1]{\textcolor{black}{\mbox{#1}}}
\newcommand{\formelbuch}[1]{$_{\textcolor{red}{\mbox{\small{S#1}}}}$}

\newcommand{\kuchling}[1]{$_{\textcolor{red}{\mbox{\small{Kuchling #1}}}}$}
\newcommand{\stoecker}[1]{$_{\textcolor{grey}{\mbox{\small{Stöcker #1}}}}$}
\newcommand{\sachs}[1]{$_{\textcolor{blue}{\mbox{\small{Sachs S. #1}}}}$}
\newcommand{\hartl}[1]{$_{\textcolor{green}{\mbox{\small{Hartl S. #1}}}}$}

\newcommand{\schaum}[1]{\tiny Schaum S. #1}

\newcommand{\skriptsection}[2]{\section{#1 {\tiny Skript S. #2}}}
\newcommand{\skriptsubsection}[2]{\subsection{#1 {\tiny Skript S. #2}}}
\newcommand{\skriptsubsubsection}[2]{\subsubsection{#1 {\tiny Skript S. #2}}}

\newcommand{\matlab}[1]{\footnotesize{(Matlab: \texttt{#1})}\normalsize{}}

% Syntax: \bmu{Pfad zum Bild}{Bildgrösse}{Beschriftung des Bildes}
\newcommand{\bl}[2]{
	\begin{figure}[h]
		\flushleft  % linksbuendig
		\includegraphics[width=#1]{#2} \\
	\end{figure}
}
\newcommand{\br}[2]{
	\begin{figure}[h]
		\flushright  % rechtsbuendig
		\includegraphics[width=#1]{#2} \\
	\end{figure}
}

\newcommand{\bild}[2]{
	\begin{figure}[h]
		\centering  % zentriert
		\includegraphics[width=#1]{#2} \\
	\end{figure}
}

\newcommand\tabbild[2][]{%
	\raisebox{0pt}[\dimexpr\totalheight+\dp\strutbox\relax][\dp\strutbox]{%
		\includegraphics[#1]{#2}%
	}%
}

\newcolumntype{P}[1]{>{\raggedright\arraybackslash}p{#1}} %Tabelle linksausgerichtet
\newcolumntype{L}[1]{>{\raggedleft\arraybackslash}p{#1}} %Tabelle rechtsausgerichtet
\newcolumntype{C}[1]{>{\centering\arraybackslash}p{#1}}



%%%%%%%%%%
% Farben %
%%%%%%%%%%
\definecolor{black}{rgb}{0,0,0}
\definecolor{red}{rgb}{1,0,0}
\definecolor{white}{rgb}{1,1,1}
\definecolor{grey}{rgb}{0.8,0.8,0.8}
\definecolor{green}{rgb}{0,.8,0.05}
\definecolor{brown}{rgb}{0.603,0,0}
\definecolor{mymauve}{rgb}{0.58,0,0.82}


%%%%%%%%%%%%%%%%%%%%%%%%%%%%
% Mathematische Operatoren %
%%%%%%%%%%%%%%%%%%%%%%%%%%%%
\DeclareMathOperator{\sinc}{sinc}
\DeclareMathOperator{\sgn}{sgn}
\DeclareMathOperator{\Real}{Re}
\DeclareMathOperator{\Imag}{Im}
%\DeclareMathOperator{\e}{e}
\DeclareMathOperator{\cov}{cov}
\DeclareMathOperator{\PolyGrad}{PolyGrad}

%Grösse Integral anpassen
\def\Int{\mbox{\Large$\displaystyle\int$\normalsize}}
\def\OInt{\mbox{\Large$\displaystyle\oint$\normalsize}}

%Makro für 'd' von Integral- und Differentialgleichungen 
\newcommand*{\diff}{\mathop{}\!\mathrm{d}}

%%%%%%%%%%%%%%%%%%%%%%%%%%%
% Fouriertransformationen %
%%%%%%%%%%%%%%%%%%%%%%%%%%%

% Fouriertransformationen
\unitlength1cm
\newcommand{\FT}
{
	\begin{picture}(1,0.5)
	\put(0.2,0.1){\circle{0.14}}\put(0.27,0.1){\line(1,0){0.5}}\put(0.77,0.1){\circle*{0.14}}
	\end{picture}
}


\newcommand{\IFT}
{
	\begin{picture}(1,0.5)
	\put(0.2,0.1){\circle*{0.14}}\put(0.27,0.1){\line(1,0){0.45}}\put(0.77,0.1){\circle{0.14}}
	\end{picture}
}


%%%%%%%%%%%%%%%%%%%%%%%%%%%%
% Allgemeine Einstellungen %
%%%%%%%%%%%%%%%%%%%%%%%%%%%%

%Pdf Info
\hypersetup{pdfauthor={\authorname},pdftitle={\titleinfo},colorlinks=false}
\author{\authorname}
\title{\titleinfo}

% Abstände Text zu Übertiteln / Einzug
\titlespacing{\section}{12pt}{1em}{0.5em}
\titlespacing{\subsection}{12pt}{1em}{0.5em}
\titlespacing{\subsubsection}{12pt}{1em}{0.5em}

%%%%%%%%%%%%%%%%%%%%%%%
% Kopf- und Fusszeile %
%%%%%%%%%%%%%%%%%%%%%%%
\pagestyle{fancy}
\fancyhf{}
%Linien oben und unten
\renewcommand{\headrulewidth}{0.5pt} 
\renewcommand{\footrulewidth}{0.5pt}

%Kopfzeile links bzw innen
\fancyhead[L]{\titleinfo{ }\tiny{(\versioninfo)}}
%Kopfzeile mitte
%\fancyhead[C]{}
%Kopfzeile rechts bzw. aussen
\fancyhead[R]{Seite \thepage { }von \pageref{LastPage}}

%Fusszeile links bzw. innen
\fancyfoot[L]{\footnotesize{\authorname}}
%Fusszeile mitte
%\fancyfoot[C]{\footnotesize{\authoremail}}
%Fusszeile rechts bzw. ausen
\fancyfoot[R]{\footnotesize{\today}}
% Einrücken verhindern versuchen
\setlength{\parindent}{0pt}

%%%%%%%%%%%%%%%%%%%%%%%%%%%%%%%%%%%%%%%
%% Makros & anderer Low-Level bastel %%
%%%%%%%%%%%%%%%%%%%%%%%%%%%%%%%%%%%%%%%
% Zeilenhöhe Tabellen:
\newcommand{\arraystretchOriginal}{1.5}
\renewcommand{\arraystretch}{\arraystretchOriginal}

\makeatletter
%% Makros für den Arraystretch (bei uns meist in Tabellen genutzt, welche Formeln enthalten)
% Default Value
\def\@ArrayStretchDefault{1} % Entspricht der Voreinstellung von Latex

% Setzt einen neuen Wert für den arraystretch
\newcommand{\setArrayStretch}[1]{\renewcommand{\arraystretch}{#1}}

% Setzt den arraystretch zurück auf den default wert
\newcommand{\resetArrayStretch}{\renewcommand{\arraystretch}{\@ArrayStretchDefault}}

% Makro zum setzten des Default arraystretch. Kann nur in der Präambel verwendet werden.
\newcommand{\setDefaultArrayStretch}[1]{%
    \AtBeginDocument{%
        \def\@ArrayStretchDefault{#1}
        \renewcommand{\arraystretch}{#1}
    }
}
\makeatother

% Settings which are used to set the distance above and under the sections
%\titlespacing*{\paragraph}{0pt}{2.25ex plus 1ex minus .2ex}{1.0ex plus .2ex}
\titlespacing{\section}{0em}{0.5em}{0.5em}
\titlespacing{\subsection}{0em}{0.5em}{0.5em}
\titlespacing{\subsubsection}{0em}{0.5em}{0.5em}
% Linksb�ndig
\setlength\parindent{0ex}

%To delete the whitespace before an Itemize
\setlist[itemize]{noitemsep, topsep=0pt}
%% Achtung Symbol \danger
\newcommand*{\TakeFourierOrnament}[1]{{%
        \fontencoding{U}\fontfamily{futs}\selectfont\char#1}}
\newcommand*{\danger}{\TakeFourierOrnament{66}}
%\usepackage{circuitikz}
% Möglichst keine Ergänzungen hier, sondern in header.tex

%%%%%%%%%%%%%%%%%%%%%%%%%%%%%%%%%%%%%%%%%%%%%%%%%%%%%%%%%%%%%%%%%%%%%%%%%%%%%%%%%%%%%%%%%%%%%%%%
%%%%%%%%%%%%%%%%%%%%%%%%%%%%%%%%%%%%%%%%%%%%%%%%%%%%%%%%%%%%%%%%%%%%%%%%%%%%%%%%%%%%%%%%%%%%%%%%

\begin{document}
\maketitle
\tableofcontents
\thispagestyle{empty}
\newpage
\section{V1}
\subsection{Anwendung und Grundlage der uP-Technik}

\begin{minipage}{8cm}
    \subsection{Aufbau}
    Verstehe die wesentlichen Systemkomponetnen des Rechnersystems auf einem IC (Integrated Circuit)
\end{minipage}
\begin{minipage}{0.5\linewidth}
    \includegraphics[width=\linewidth]{images/aufbauuC}
\end{minipage}

\begin{multicols}{2}
\subsubsection{Anwendungen}
\begin{minipage}{\linewidth}
\begin{itemize}
    \item Supercomputer
    \item Arbeits und Server-Rechnern
    \item Smartphones
    \item Navigationssysteme
    \item Digitalkameras
    \item Drucker
    \item ...
\end{itemize}
\end{minipage}

\begin{minipage}{\linewidth}
\subsubsection{Aufbau von uP-basierten Systemen}
\begin{itemize}
    \item Zentraleinheit CPU mit
    \begin{itemize}
        \item Rechenwerk ALU
        \item Steuerwerk CU
        \item Registersatz
    \end{itemize}
    \item Speicher
    \item Eingabe-/Ausgabe-Schnittsellen
\end{itemize}
\end{minipage}
\end{multicols}

\includegraphics[width=0.5\linewidth]{images/aufbauuC1}
\includegraphics[width=0.5\linewidth]{images/aufbauuCspeicher}

\subsubsection{Havard vs Von Neumann Architektur}
\begin{multicols}{2}
    \begin{minipage}{\linewidth}
        \textbf{Harvard Rechnermodell}\\
        \includegraphics[width=0.6\linewidth]{images/HavardArchi}
    \end{minipage}
    
    \begin{minipage}{\linewidth}
        \textbf{von Neumann Rechnermodell}\\
        \includegraphics[width=0.6\linewidth]{images/NeumannArchi}
    \end{minipage}
\end{multicols}
\clearpage
%===================================
\subsubsection{Programmierung eins uP}
\begin{multicols}{2}
\begin{minipage}{\linewidth}
    Ein \mu P kann durch individuelle Programmierung auf ganz unterschiedliche Art angepasst werden. \newline
    \rightarrow entscheidend für die Durchdringung im Markt.\newline
    Ein Programm enthält in aufeinanderfolgender Anordnung die Maschinen-Befehle oder -Instruktionen für den \mu P. Diese Maschiene-Befehle teilen der CPU mit, welche Operationen in welcher Reihenfolge und auf welche Daten angewendet werden sollen. \newline
    Die Befehlsfolge des Programms wird innerhalb der CPU vom Steuerwerk gesteuert und schrittweise ausgeführt. Dazu wird der aktuell zur bearbeitende Befehl durch einen Programmzähler(PC) im Speicher adressiert.\newline
    Der PC enthält laufend die Adresse der Speicherzelle des jeweiligen Befehls im Speicher.
\end{minipage}

\includegraphics[width=\linewidth]{images/uPPC}
\end{multicols}
    
\subsubsection{Befehlsformate}
\begin{multicols}{2}
\begin{minipage}{\linewidth}
Die Art und Wirkung eines Befehls wird im Befehlswort(\textbf{OpCode}) codiert.
Darin sind neben der Operation auch die Operanden spezifiziert.
Die Codierung des Befehlswortes erfolgt abhängig vom \mu P.
Der Maschinencode setzt sich aus einem OpCode und einem oder mehreren Operanden zusammen.
\end{minipage}

\includegraphics[width=\linewidth]{images/Befehlsformate}
\end{multicols}


\subsection{RISC vs CISC}
\begin{multicols}{2}
    \textbf{CISC}\newline
    Complex Instruction Set Computer
    \\
    \textbf{RISC}\newline
    Reduced Instruction Set Computer
\end{multicols}
\subsubsection{RISC-Rechner}
effizienter als CISC-Rechner
\begin{itemize}
    \item besteht aus einer kleien Anz. von Befehlen mit wenigen Adressierungsarten
    \item Registersatz enthält eine grosse Anzahl von allg. verwendbaren Registern\newline
    General Purpose Register (GPR)
    \item Speicherzugriff erfolgt über spezielle Lade- und Speicher-Befehle
    \begin{itemize}
        \item Arithmetische-logische Operationen arbeiten auf Registeroperanden
    \end{itemize}
    \item Pipline-Architecture \leftarrow Leistungssteigernde Architektur
    \item Eine grosse semantische Lücke entsteht bei der Übersetzung aus der Hochsprache
\end{itemize}

\subsubsection{u Architektur}
Beschriebt die architektonischen Details bei der Implementierung der \mu P aus Sicht der Programmierer.
Dies umfassst die Beschreibung der Zentraleinheit (CPU), des Rechenwerks (ALU) und des Steuerwerks (CU).



\begin{minipage}{10cm}
\subsection{Hardware}
\subsubsection{Registersatz}
Register sind schnelle Zwischenspeicher für \newline
temporäre Daten im \mu P.
\end{minipage}
\begin{minipage}{0.5\linewidth}
\subsubsection{Hardware- /Software-Schnitsttelle}
\includegraphics{images/HardwareSoftware}
\end{minipage}

\subsubsection{Taktfrequenz}
Das Taktsignal steuert die zeitliche Abfolge im \mu P \newline
\begin{multicols}{2}
        \begin{minipage}{\linewidth}
    \[ f_{Takt}= \frac{1}{T_{takt}} \]
        \end{minipage}
    
    \begin{minipage}{\linewidth}
        $ \Uparrow $ Taktrate $ \Leftrightarrow $$  \Uparrow  $Leistungsaufnahme \\
        Um Energie zu sparen ist es sinnvoll die Taktrate laufend anzupassen.\\
    \end{minipage}
\end{multicols}
\subsubsection{Leistungsaufnahme}
\begin{multicols}{2}
        \begin{minipage}{\linewidth}
\[ P_{Gate}= \frac{1}{2} \cdot C_{Last} \cdot V_{DD}^2\cdot f_{Takt} \]
    \end{minipage}
    
\begin{minipage}{\linewidth}
    $ P_{Gate} \qquad $Leistung pro CMOS Gate \newline
    $ C_{Last} \qquad $Lastkapazität\newline
    $ V_{DD}   \qquad $Versorgungsspannung\newline
    $ f_{Takt} \qquad $Taktfrequenz
\end{minipage}
\end{multicols}

\subsection{Software}
\begin{multicols}{2}
\subsubsection{Ablauf}
\begin{itemize}
    \item Der Präprozessor bereitet das Quellprogramm für den Compiler vor
    \item Der Compiler übersetz das Programm von einer Hochsprache in Assembly-Programm
    \item Der Binder fasst verschiedene Dateien, die verschiebbaren Maschinencode enthalten, zu einem Programm zusammen.
    \item Der Loader wandelt die verschiebbaren Adressen in Absolute Adressen um und läd sie in den Speicher des Systems.
\end{itemize}

\includegraphics[width=\linewidth]{images/CompilerWorkflow}
\end{multicols}
\clearpage





















    
\clearpage
\section{V2}
\subsection{Compiler-Schritte}
\vspace{-0.5cm}
\begin{multicols}{2}
    \begin{minipage}{\linewidth}
        \begin{enumerate}
            \item \textbf{Lexikalische Analyse (Scanning):}\newline
                Die Symbole der Sprache werden erkannt und Gruppiert. Leerzeichen werden eliminiert
            \item \textbf{Syntaxanalyse (Parsing):}\newline
                 Die erkannten Symbole werden in Sätzen zusammengefasst und in einem Parsbaum dargestellt
            \item \textbf{Semantische Analyse:}\newline
                 Das Quellprogramm wird auf Fehler überprüft (zBsp. Typfehler) und der Parsbaum erhält Informationen über die verwendeten Bezeichner
            \item \textbf{Zwischencode-Erzeugung:}\newline
                 Einige Compiler erzeugen Code in einer Zwischensprache (abstrakte Maschinen)
            \item \textbf{Code-Erzeugung:}\newline
                 Erzeugen von verschiebbarem Maschinencode.      
        \end{enumerate}
    \end{minipage}

    \includegraphics[width=\linewidth]{images/CompilerWorkflow2}
\end{multicols}

\subsection{Busorientierte Systeme}
\subsubsection{Speicher}
\begin{multicols}{2}
    \textbf{RAM}
    \begin{itemize}
        \item Random Access Memory
        \item Schreibe-/Lese-Speicher
        \item Spannungsversorgung erforderlich
        \item für temporäre Daten
    \end{itemize}

    \textbf{ROM}
    \begin{itemize}
        \item Read Only Memory
        \item Festwert Speicher
        \item auch ohne Spg. bleiben Daten erhalten
    \end{itemize}
\end{multicols}

\begin{multicols}{3}
    \begin{minipage}{4cm}
        \textbf{Adressbus}
        \begin{itemize}
            \item unidirektional
            \item bestimmt Grösse des Adressraums
        \end{itemize}
    \end{minipage}
    
    \begin{minipage}{4cm}
        \textbf{Databus}
        \begin{itemize}
            \item bidirektional
        \end{itemize}
    \end{minipage}
    
    \begin{minipage}{5cm}
        \textbf{Steuerbus}
        \begin{itemize}
            \item kontrolliert Buszugriffe
            \item zeitlicher Ablauf der Signale
        \end{itemize}
    \end{minipage}
\end{multicols}

\begin{itemize}
    \item[*] Die gesammte Menge der über den Adressbus adressierbaren Speicherzellen wird \textbf{Adressraum} genannt
    \item[*] Die Anzahl parallel geführten \textbf{Datenleitungen} entspricht der maximal zu übertragenden Datenbreite
    \item[*] Kontrollsignale werden über den \textbf{Steuerbus} übertragen
\end{itemize}
    \includegraphics[width=0.5\linewidth]{images/SimpleuPSystem}
    
\subsubsection{Architectur eines uP}
    \begin{minipage}{12cm}
        \includegraphics[width= 12cm]{images/ArchitectuP}
    \end{minipage}
	%
	\begin{minipage}{0.5cm}
		\ \
	\end{minipage}
	%
    \begin{minipage}[r]{4cm} 
        \begin{tabular}[r]{c|c}
            \textbf{AR} & Adressregister  \\ 
            \textbf{PC} & Programm Counter  \\ 
            \textbf{PSR}& Program Status Register \\ 
            \textbf{IR} & Instruction Register  \\ 
        \end{tabular} 
	\end{minipage}
    
	\begin{minipage}{9cm}
        \textbf{Flags}, welche sich unter anderem im PSR befinden\newline
        \begin{tabular}{|c|c|l|}
            \hline 
            \textbf{N}  &\textbf{Negative}  & Das von der ALU berechnete Ergebniss ist negativ \\ 
            \hline 
            \textbf{Z}  &\textbf{Zero}      & Das von der ALU berechnete Ergenis ist gleich \textbf{0} \\ 
            \hline 
            \textbf{C}  &\textbf{Carry}     & Die Berechnung der ALU hat zu einem Übertrag geführt  \\ 
            \hline 
            \textbf{V}  &\textbf{Overflow}  & Die Berechnung der ALU hat zu einem Overflow geführt  \\ 
            \hline 
        \end{tabular} 
	\end{minipage}

\subsection{Befehlszyklus}
    \includegraphics[height=9cm]{images/CommandFlowChart}

\clearpage
\section{V3}
\vspace{-0.5cm} 
    \begin{minipage}{9cm}
        \subsection{Halbleiter Speicher} 
        \textbf{Zentraler Speicher}
        \begin{itemize}
            \item direkt am Bussystem angeschlossen
        \end{itemize}
        \textbf{Peripherer Speicher}
        \begin{itemize}
            \item über I/O-Schnittstelle angeschlossen
        \end{itemize}
    \end{minipage}
    %
    \begin{minipage}{0.5cm}
    	\ \
    \end{minipage}
    %
    \begin{minipage}{9cm}
    	\includegraphics[width=9cm]{images/halbleiterfam}
    \end{minipage}
 
\begin{multicols}{2}
    \subsubsection{ROM-Festwertspeicher}
    \includegraphics[width=8cm]{images/ROM}
    
    \subsubsection{RAM-Speicher-/Lese-Speicher}
    \includegraphics[width=8cm]{images/RAM}
\end{multicols}

\subsection{Speicherorganisation}
\begin{minipage}{9cm}
	\subsubsection{Little/Big Endian}
    \includegraphics[width=8cm]{images/LittleBigEndian}
    
    Der ARM Cortex verwendet standardm"assig \textbf{Little Endian} f"ur die Speicherorganisation. 
\end{minipage}
%
\begin{minipage}{0.5cm}
	\-\
\end{minipage}
%
\begin{minipage}{9cm}
	 \subsubsection{I/O - Schnittstelle}
    \includegraphics[width=8cm]{images/IOSchnittstelle}
    
    \textbf{Datenregister}
    \begin{itemize}
    	\item enthalten die zu verarbeitenden Daten
    \end{itemize}
    \textbf{Steuerregister}
    \begin{itemize}
    	\item dienen zur Konfiguration der Ein-/Ausgabe Schnittstelle
    \end{itemize}
    \textbf{Statusregister}
    \begin{itemize}
    	\item signalisieren den Zustand der Ein-/Ausgabe Schnittstelle
    \end{itemize}
\end{minipage}
 
\subsubsection{Speicherraumadressierung}
\begin{minipage}{11cm}
	\includegraphics[width=11cm]{images/Speicherraumadressierung}
\end{minipage}
%
\begin{minipage}{0.5cm}
	\-\
\end{minipage}
%
\begin{minipage}{7cm}
	\textbf{Isolierte Adressierung}
	\begin{itemize}
		\item Gesamter Adressraum steht verschiedener Bl"ocken zur Verf"ugung
		\item Bl"ocke werden mit einem Steuersignal ausgew"ahlt
	\end{itemize}
	\textbf{Memory-Mapped I/O (Cortex M3)}
	\begin{itemize}
		\item Verschiedene Bl"ocke werden in einen Adressraum eingebettet
		\item Ben"otigt eine Adresskodierung
	\end{itemize}
\end{minipage}

\newpage
\subsection{Busanschluss und Adressverwaltung}
Eine Adressverwaltung hat verschiedene Aufgaben zu erf"ullen:
\begin{itemize}
	\item Jede Adresse spricht nur einen einzigen Speicher- oder I/O-Baustein an.
	\item Adressraum m"oglichst gut ausn"utzen
	\item Adressr"aume von Speicherbausteinen m"ussen l"uckenlos aufeinander folgen.
	\item Jeder interne Speicherplatz bzw. jedes Register erscheint unter einer eigenen Adresse im Systemadressraum.
\end{itemize}

\subsubsection{Adresskodierung}
\includegraphics[width=14cm]{images/Adressverwaltung}\\
Der Kernpunkt der Adresskodierung ist die Teilung des Adressbuses. Dabei werden die \textit{k} niedrigsten Adressleitungen direkt an die Adresseing"ange des Bausteines gef"uhrt und dienen zur Auswahl des gew"unschten internen Speicherplatzes oder Registers. Die n"achstfolgenden \textit{l} Adressleitungen werden zur Adresskodierung auf einen Adressdekoder gef"uhrt.
\clearpage
%============================================























\clearpage
\section{V4}

\subsection{Cortex-M3/M4}
\begin{itemize}
    \item Harvard Architecture
    \subitem \rightarrow Zugriffe auf Instruktionen und Daten können gleichzeitig stattfinden
    \item Internal Bus Interconnect
    \subitem \rightarrow mehrere Bus-Interface 
    \item Nested Interrupt Controller \textbf{(NVIC)}
    \item Standart Timer \textbf{(SYSTICK)}\\
    \textbf{Optional:}
    \item Memory Protection Unit \textbf{(MPU)}
    \item Floating Point Unit \textbf{(FPU)}
\end{itemize}
\subsection{System-Komponenten}
\begin{multicols}{2}
\subsubsection{NVIC}
\begin{itemize}
    \item Non-Maskable Interrupt (NMI)
    \item Bis zu 240 externe Interrupts
    \item 8 bis 256 Prioritätslevel
\end{itemize}
\rightarrow ISR benötigt 12 Taktzyklen
\subsubsection{FPU - (nur Cortex M4!)}
Mit der FPU lassen sich IEEE754 Signal Precision Floating-Point Operationen in sehr wenigen Takten ausführen

\subsubsection{SYSTICK}
\begin{itemize}
    \item 24-Bit Countdown-Timer mit automatischer Relaod-Funktion
    \item Wird für einen periodischen Interrupt verwendet
\end{itemize}
Wenn der Zähler den Wert 0x000000, wird dies dem NVIC signalisiert und  der Reload-Wert wird aus dem Reload-Register gelesen.
\includegraphics[width=6cm]{images/NVIC}
\end{multicols}

\begin{multicols}{2}
    \begin{minipage}{\linewidth}
        \subsubsection{WIC (Wakeup Interrupt Controller)}
        Für die Umsetzung von Low-Power-Modes\newline
        Dadruch kann 99\% der Cortex M3-Prozessoren im Low-Power-Bereich arbeiten.
        \newline
        Ist mit dem NVIC verknüpft und holt den Prozessor aus diesem Modus heraus, um auf einen Interrupt reagieren zu können
    \end{minipage}
    
    \begin{minipage}{\linewidth}
        \subsubsection{MPU}
        \begin{itemize}
            \item ermöglicht Zugriffsregel für den Privilieged Access und User Programm Access zu definieren
            \item \rightarrow Wird eine Zugriffsrege verletzt erfolgt eine Exception-Regelung wodurch der Exceptrion Handler das Problem analysiert und ggf. beheben kann
            \item \rightarrow Ausserdem ist es möglich gewisse Bereiche als read-only zu deklarieren
            \end{itemize}
    \end{minipage}
\end{multicols}
\clearpage
\subsection{GNU-Tool-Chain Entwicklungsablauf}
\begin{multicols}{2}
          \begin{minipage}{\linewidth}
    \includegraphics[width=\textwidth]{images/gnutoolchain}
    \begin{tabular}{|l|l|}
        \hline 
        \textbf{APSR}& Application Program Status Register \\ 
        \hline 
        \textbf{IPSR}& Interrupt Program Status Register \\ 
        \hline 
        \textbf{EPSR}& Execution Program Status Register \\ 
        \hline 
    \end{tabular}
    \subsubsection{SP-zugriffe(Assembler)}
      \includegraphics[width=\textwidth]{images/SPzugriffe}   
\end{minipage}
    
    \includegraphics[width=0.5\textwidth]{images/gnutoolchain1}
\end{multicols}

\subsection{Programm Status Register}
\begin{minipage}{\linewidth}
    \begin{tabular}{|l|l|}
        \hline 
        \textbf{N}& Negativ \\ 
        \hline 
        \textbf{Z}& Zero  \\ 
        \hline 
        \textbf{C}& Carry/borrow  \\ 
        \hline 
        \textbf{V}& Overflow \\ 
        \hline 
        \textbf{Q}& Sticky saturation flag \\ 
        \hline 
        \textbf{ICI/IT}& Interrupt-Continauble Instruction(ICE) bits\\
                        & IF-THEN instruction status bit \\ 
        \hline 
        \textbf{T}& Thumb state, always 1; Trying to clear this bit will caus a fault exception \\ 
        \hline 
        \textbf{Exception number}& INdicates whiche exception the processor is handeling \\ 
        \hline 
    \end{tabular} 
\end{minipage}

\includegraphics[width=17cm]{images/programstatusregister}
\clearpage

\subsection{Stack}
\begin{multicols}{2}
\begin{minipage}{0.5\textwidth}
    \begin{itemize}
        \item Temporäre Zwischenspeicherung von Daten während der ausführung einer Funktion
        \item Übergabe von Informationen an Funktionen oder Subroutinen
        \item Speichern von lokalen Variabeln
        \item Erhalten von Prozessor-Status und Register-Werten, während Exceptions oder Interrupts ausgefüht werden
        \item PUSH-POP-Instruktionen werden ausgeführt
        \item LIFO-Prinzip(Last In, First Out)
    \end{itemize}
\end{minipage}

\begin{minipage}{0.5\textwidth}
    \subsubsection{Main-Stak-Pointer (MSP)}
    \begin{itemize}
        \item Standart Stack Pointer nach einem Reset
        \item Innerhalb von Exception-INterrupt-Handler wird imer der MSP benutzt!
        \end{itemize}
    \subsubsection{PRozesor-Stack-Pointer (PSP)}
    \begin{itemize}
        \item Alternativer Stackpointer
        \item Wird nur im Thread-Mode verwendet
        \subitem \rightarrow bei embedded OS-System
    \end{itemize}   
\end{minipage}
\end{multicols}


















\clearpage
\section{V5}
\subsection{Data Alignment}
\begin{minipage}{6cm}   
    \begin{tabular}{|l|l|l|}
        \hline 
        1 Byte & 8 Bit  & Byte  \\ 
        \hline 
        2 Byte & 16 Bit & Half-Word \\ 
        \hline 
        4 Byte & 32 Bit & Word \\ 
        \hline 
        8 Byte & 64 Bit & Double-Word \\ 
        \hline 
    \end{tabular} 
\end{minipage}
\begin{minipage}{14cm}
    \subsubsection{Aligned-Unaligned Data}
    \begin{multicols}{2}
            \includegraphics[width=0.5\textwidth]{images/alignedData}
            \includegraphics[width=0.5\textwidth]{images/unalignedData}
    \end{multicols}
\end{minipage}
\subsubsection{Bit-Banding}
\vspace{-0.5cm}
\[\textbf{ BitBandAliasAddress = BitBandAliasBase + (MemoryAddres - BitbandRegionBase)* 32 + 4*BitNumber} \]
\[ BNr=[mod_{32}(BBAA-BBAB)]\cdot 2^2 \quad \rightarrow mod_{32}\; =\; 5\; Stellen\; von\: LSB\: in\: binär \]
\[ MA = (BBAA-BBAB)\cdot 2^{-5} +BBRB \]
\includegraphics[width=0.8\textwidth]{images/bitbanding}
\clearpage
\section{V6}
\subsection{Exceptions and Interrupts}\label{Exceptions}
\begin{minipage}{10cm}
	Exceptions sind Ereignisse, die den sequenziellen Programmablauf ver"andern. Der Prozessor unterbricht den normal laufenden Programmablauf (Background) und f"uhrt einen Exception-Handler (Foreground) aus. Exceptions werden vom NVIC (nested vectored interrupt controller) verarbeitet. Der NVIC kann eine Reihe von  \textit{Interrupt Request (IRQs)} und einen \textit{Non-Maskable Interrupt (NMI)} verarbeiten, wobei der \textit{NMI} beispielsweise von einem Watchdog-Timer aktiviert werden kann. 
\end{minipage}
%
\begin{minipage}{0.5cm}
	\-\
\end{minipage}
%
\begin{minipage}{9cm}
	\includegraphics[width=8.5cm]{images/NVICExcp}
\end{minipage}

Der Prozessor selbest ist auch eine Quelle von Exceptions.\\
Jede Exception-Quelle hat eine zugeh"orige Exception-Number:
        \begin{itemize}
        	\item Exception-Number 1-15 gelten als \textit{System-Exceptions}
        	\item Exception-Number 16-255 sind \textit{Interrupts}
        \end{itemize} 
        Der NVIC kann bis zu 240 IRQs verarbeiten, in der Praxis sind es aber oft weniger.
    
Exception-Handler für Interrupts werden als Interrupt Service Routine (ISR) bezeichnet, wobei die Interrupt-Latenzzeit (\textit{Interrupt Latency}) gerade mal 12 Clockzyklen betr"agt.

\subsubsection{NVIC}
Jeder Interrupt kann individuell aktiviert/deaktiviert werden, ausserdem kann sein \textit{Pending State} durch die Software gesetzt/gel"oscht werden. Den Exceptions k"onnen \textit{Priority-Level} zugeordnet werden und somit ausgel"oste ISRs mit tieferer Priorit"at durch einen Interrupt h"oherer Priorit"at unterbrochen werden. 

\begin{minipage}{10cm}
	Es gibt 8 verschiedene Priorit"atsstufen, wobei diese noch in Priorit"atsgruppen aufgeteilt werden k"onnen. Daraus entstehen dann \textit{Preemptive levels} und \textit{Subpriority levels}. Wenn nun zum Beispiel zwei Interrupts das gleiche \textit{Preemptive level} besitzen, kann mit dem \textit{Subpriority level} bestimmt werden, welcher Interrupt beim gleichzeitigen auftreten zuerst ausgef"uhrt wird. Ein Interrupt mit einer h"oheren \textit{Subpriority} kann jedoch keinen einer tiefen \textit{Subpriority} unterbrechen, wenn sie das gleiche \textit{Preemptive level} besitzen.
\end{minipage}
%
\begin{minipage}{0.25cm}
	\-\
\end{minipage}
%
\begin{minipage}{8cm}
	\includegraphics[width=8cm]{images/prioritaetsstufen}
\end{minipage}

\subsection{Reset und Reset-Sequenzen}
\subsubsection{Reset}
Es gibt 3 Arten von Reset:\\
\begin{tabular}{ll}
    \textbf{Power-on Reset}  & Resettet den gesamten $\mu$ C, auch alle Peripherien und Debug-Komponenten \\ 
    \textbf{System Reset}    & Resettet nur den Prozessor und die Peripherien, aber nicht die Debug-Komponenten \\ 
    \textbf{Processoer Reset} & Resettet nur den Prozessor\\
\end{tabular}

\subsubsection{Reset Sequenz}
\begin{minipage}{9cm}
	Nach einem Reset und bevor der Cortex-M Prozessor mit der eigentlichen Programmausf"uhrung startet, liest die CPU die ersten beiden 32-Bit Word aus dem Speicher. Am Anfang in der Vektor-Tabelle steht der \textbf{Initial} \textit{Main-Stack-Pointer (MSP)} gefolgt vom \textbf{Initial} \textit{Program Counter (PC)}. Das Setup des MSP ist notwendig, um von Beginn weg einen g"ultigen Stack zu haben.
\end{minipage}
%
\begin{minipage}{0.5cm}
	\-\
\end{minipage}
%
\begin{minipage}{9cm}
	\includegraphics[width=9cm]{images/resetsequenz}
\end{minipage}

\subsection{Fault-Handling}
\begin{minipage}{9cm}
	Der Fault-Exception Mechanismus erlaubt eine schnelle Reaktion auf Systemfehler und gibt der Software die M"oglichkeit, Notfallszenarien einzuleiten. Standardm"assig sind die Exceptions \textit{Bus Fault, Usage Fault} und \textit{Memory Management Fault} deaktiviert, stattdessen triggern alle drei Faults die \textit{Hard Fault Exception}.
\end{minipage}
%
\begin{minipage}{0.5cm}
	\-\
\end{minipage}
%
\begin{minipage}{9cm}
	\includegraphics[width=9cm]{images/fault-handling}
\end{minipage}
    
\subsection{Spezial-Register}     
Register um Exceptions ein oder auszuschalten:\\
\textbf{$\rightarrow$ PRIMASK,FAULTMASK,BASEPRI}

\begin{minipage}{9cm}
    \subsubsection{PRIMASK}
    \begin{itemize}
        \item 1-bit Register
        \item Wenn das aktiv ist, werden NMI-Interrupts erlaubt
        \subitem $\rightarrow$ alle anderen Interrupts werden überdeckt
        \subitem $\rightarrow$ Default-Wert= 0, also deaktiviert
    \end{itemize}
 \end{minipage}
 %
 \begin{minipage}{0.5cm}
 	\-\
 \end{minipage}  
 %
 \begin{minipage}{9cm}
 	\subsubsection{FAULTMASK}
    \begin{itemize}
        \item 1-bit Register
        \item Wenn das aktiv ist, werden nur noch NMI-Interrups akzeptiert.\newline
        Alle anderen Interrupts und Exceptioln-Handlings werden deaktiviert
        \subitem $\rightarrow$ Default-Wert = 0
    \end{itemize}
 \end{minipage}

\subsubsection{BASEPRI}
\begin{itemize}
    \item Wenn das gesetzt wird, werden alle Interrupts mit gleicher oder tieferer Stufe deaktiviert
\end{itemize}

\subsubsection{Control-Register}
Das Kontroll-Register definiert:
\begin{enumerate}
    \item Die Auswahl zwischen MSP (Main-SP) und PSP (Process-SP)
    \item Die Zugriffsstufe und Thread-Mode
    \subitem (Ob Privilegd oder unprivilegd)
\end{enumerate}
\begin{multicols}{2}
    \includegraphics[width=\linewidth]{images/StackPointerAuswahl}
    
    \includegraphics[width=\linewidth]{images/controlRegister}
\end{multicols}
\clearpage
\section{V7}
\subsection{Cortex M3 Instruction Set}
\begin{multicols}{2}  
\subsubsection{Thumb-2 Instruction Set}
\begin{tabular}{l l}
    Ziel&-Erhöht die Code-Dichte\\
        &  -Mehr Leistung\\ 
  Cortex M3 Processor  & 1.25 DMIPS / MHz\\ 
\end{tabular} 

\begin{minipage}{\textwidth}
\subsection{Logikstruktur des Cortex-M Prozessor}
\begin{tabular}{ll} 
    Sourceoperanden& Rn, Rm \\ 
    Destinationsoperand& Rd  \\ 
\end{tabular} \\
Ein \textit{Barrel-Shifter} vereinfacht Berechnungen,\newline
da Multiplikationen einfacher realisiert werden können.
\end{minipage}
\end{multicols}

\begin{multicols}{2}
     \subsection{Instruction Pipelining}
     \begin{minipage}{\linewidth}
         MAC = Memory Access Calculator\\
         Load Store Architektur
     \end{minipage}   
     \includegraphics[width=\linewidth]{images/pipelining}
     
    \includegraphics[width=0.8\linewidth]{images/logikstrukturcortex}
\end{multicols}

\subsection{Anwendungen}
\begin{multicols}{3}
    \begin{minipage}{\linewidth}
       \subsubsection{Cortex-M0/M0+ / M1}  
       einfaches I/O Handling 
    \end{minipage}
    
    \begin{minipage}{\linewidth}
        \subsubsection{Cortex-M3}   
        Komplexe Datenverarbeitung\newline
        anspruchsvolle Applikationenen
    \end{minipage}

    \begin{minipage}{\linewidth}
        \subsubsection{Cortex-M4}   
        DSP-Funktionalität\newline
        Floating Point Support
    \end{minipage}
\end{multicols}
    
\subsection{Assembly-Language Syntax}
\begin{tabular}{llll}
    \textbf{Lable}  &\textbf{OpCode}  &\textbf{Operand}  & \textbf{Comment} \\ 
    L1& ADD &R0,R1,\#5  & Replace R0 by sum of R1 and 5 \\ 
    FUNC& MOV &R0,\#100  & this sets R0 to value 100 \\  
    &BX&LR& this is a function return\\
\end{tabular} 
\begin{tabular}{|ll}
    \textbf{Lable}& optional  \\ 
    \textbf{OpCode}& spezifiziert den Befehl \\ 
    \textbf{Operand}& Parameter  \\ 
    \textbf{Comment}&  optionale Beschriebung\\ 
\end{tabular} 
\\
\begin{multicols}{2}
    \begin{minipage}{\linewidth}
        \subsection{Unified Assembler Language (UAL)}
        Syntax für ARM und Thumb Instructionen.\\
        Die meisten Instruktionen arbeiten mit Registern\\
        \textbf{BSP}\newline
            \begin{tabular}{lll}
                MOV&R2,\#100  &;R2=100,Direkte Zuweisung  \\ 
                LDR&R2,[R1]  &;R2= den Wert von R1  \\ 
                ADD&R2,R0    &;R2=R2+R0  \\ 
                ADD&R2,R0,R  &;R2=R0+R1  \\ 
            \end{tabular} 
    \end{minipage}
    
    \begin{minipage}{0.8\linewidth}
        \subsubsection{Register List}
        \begin{tabular}{lll}
            Norm. Form&{reglist}  &;{R1,R2...Rn}  \\ 
            PUSH& {LR} & ;save LR on stack\\ 
            POP&  {LR}&  ;remove from stack; place in LR\\  
            PUSH& {R1-R3,LR} & ;save R1,R2,R3; return address\\  
            POP& {R1-R3,PC} &;restore R1,R2,R3 and return \\ 
        \end{tabular} 
    \end{minipage}
\end{multicols}
\clearpage

\subsection{Addressing}
\begin{multicols}{2}
    \begin{minipage}{\linewidth}
    \subsubsection{Immediate Adressing}
       Der Datenwert ist unmittelbar in der Instruktion erhalten. Daher kein zusätzlicher Speicherzugriff erforderlich\newline
       Form: \# imm\newline
       \begin{tabular}{lll}
          MOV & R0,\# 100&;R0=100, immediate addressing \\ 
        \end{tabular} 
    \end{minipage}
    \includegraphics[width=0.9\linewidth]{images/immediateAddressing}    
\end{multicols}   


\begin{multicols}{2}
    \begin{minipage}{\linewidth}
    \subsubsection{Indirect Addressing}
        Bei der indirekten Adressierung sind mehrere Speicherzugriffe erforderlich.\newline
        Form: [Rn]\newline
           \begin{tabular}{lll}
              LDR & R0,[R1]&;R0=value pointed to by R1 \\ 
            \end{tabular} \\
        Ein Register enthält irgendwie eien Zeiger auf dieses Register\newline
        \textbf{R1 wird nicht verändert}
     \end{minipage}
     \includegraphics[width=0.9\linewidth]{images/indirectAddressing}    
\end{multicols} 


\begin{multicols}{2}
    \begin{minipage}{\linewidth}
    \subsubsection{Register Addressing with Displacment}
        Dasselbe nur wierd hier dem Wert R0 noch \# 4 hinzugefügt\newline
        R1 bleibt weiterhin unverändert.\newline
        Form: [Rn,\# imm]\newline
        \begin{tabular}{lll}
            LDR & R0,[R1,\# 4]&;R0=word pointed to by R1+4 \\ 
        \end{tabular} \\
    \end{minipage}
    \includegraphics[width=0.9\linewidth]{images/AddressingDisplacment}    
\end{multicols} 

\subsubsection{Register Indirect with Index}
Form: [Rn,Rm]\newline
\begin{tabular}{lll}
   LDR &R0,[R1,R2]  &;R0= word pointed to by R1+R2 \\ 
\end{tabular} 

\subsubsection{Register Indirect with shifted Index}
Form: [Rn,Rm,LSL \# imm]\newline
\begin{tabular}{lll}
    LDR&R0,[R1,R2;LSL \#2]  &;R0= word pointed to by R1+4*R2  \\ 
\end{tabular} 

\subsubsection{Register Indirect with Pre-index}
Form: [Rn,\# offset]!\newline
\begin{tabular}{lll}
   LDR & R0,[R1,\#4]! &;first R1=R1+4, then R0= word pointed to by R1  \\ 
\end{tabular} 

\subsubsection{Register Indirect with Post-index}
Form: [Rn],\# offset\newline
\begin{tabular}{lll}
    LDR& R0,[R1],\#4  &;R0= word pointed to by R1, then R1=R1+4  \\ 
\end{tabular} 

\subsubsection{PC-relativ}
PC wird als Pointer verwendet.
Form: lable\newline
\begin{tabular}{lll}
    B   &Location   &;jump to Location\\ 
    BL  &Subroutine &;call Subroutine, Rücksprungadresse wird gespeichert\\ 
\end{tabular} 

\subsubsection{Speicher- und I/O-Zugriffe}
\begin{multicols}{2}
    \begin{minipage}{\linewidth}
    Es benötigt immer zwei Instruktionen um auf Daren im RAM oder I/O zuzugreifen.
    \rightarrow PC-Relative Addressierung wird verwendet
    \begin{enumerate}
        \item Erstellt Zeuger auf das Objekt
        \item Greift über den Zeiger Indirekt auf den Speicher zu
    \end{enumerate}
    \begin{tabular}{lll}
        LDR   &R1,Count   &;R1 points to variable Count\\ 
        LDR  &R0,[R1] &;R0= value pointed to by R1\\ 
    \end{tabular} 
\end{minipage}

    \includegraphics[width=\linewidth]{images/AddressingRAM}   
\end{multicols}






















\clearpage
\section{V8}
\begin{minipage}[t]{9cm}
	\subsection{Interne Datenverschiebung}
F"ur die Verschiebung von Daten von Register zu Register oder der Beschreibung eines Registers mit einer Konstanten kann die Operation \textbf{\textit{MOV} }verwendet werden. Diese folgend angewendet werden: \\

\includegraphics[width=9cm]{images/MOV-Instruktion}
\end{minipage}
%
\begin{minipage}[t]{0.5cm}
	\-\
\end{minipage}
%
\begin{minipage}[t]{9cm}
	\subsection{Memory Access Instructions}
Der Cortex-M3 verf"ugt "uber viele verschiedene Instruktionen f"ur den Speicherzugriff. Dies wegen den verschiedenen Adressierungsarten und Datengr"ossen. F"ur normale Datentransfers, sind folgenden Instruktionen vorhanden:\\

\includegraphics[width=9cm]{images/LDR-Instruktion}

Mit der \textbf{\textit{LDM}} bzw. \textbf{\textit{STM}} Instruktion ist es m"oglich gerade eine ganze Registerliste zu Laden bzw. zu Speichern.
\end{minipage}

\subsection{Stack Push und Pop}
\begin{multicols}{2}
    \begin{minipage}{3cm}
        PUSH \qquad {R0}\newline
        PUSH \qquad {R1}\newline
        PUSH \qquad {R2}\newline
        POP  \qquad {R3}\newline
        POP  \qquad {R4}\newline
        POP  \qquad {R5}\newline
    \end{minipage}
    \begin{minipage}{\linewidth}
        \includegraphics[width=1.5\linewidth]{images/stackpushpop}  
    \end{minipage}
\end{multicols}
 \textbf{\textit{PUSH}} und \textbf{\textit{POP}} k"onnen auch Registerlisten mitgegeben werden, wie folgendes Beispiel zeigt:
 
 \colorbox{lightgray}{
    \begin{tabular}{lll}
        PUSH   & \{R4-R6,LR\}  & ;Save R4 to R6 an LinkRegister at the beginning of a subroutine.\\ 
        && ; LR contains the return address\\
        ...   &   & ;processing the subroutine\\ 
        POP & \{R4-R6,PC\} & ;POP R4 to R6, and return address from stack. The return address\\
        && ;is stored into PC directly, this triggers a branch (subroutine return)
    \end{tabular} }


\subsubsection{Generelle Regeln bei der Verwendung des Stacks}
\begin{minipage}{9cm}
	\begin{enumerate}
        \item Funktionen sollten die gleiche Anzahl Push und Pop Befehle aufweisen.
        \item Stackzugriff nur innerhalb des allozierten Bereichs
        \item Es sollte nicht über den SP auf den Stack geschrieben oder gelesen werden.
        \item Stack sollte zuerst den SP dekrementieren und erst dann die Daten ablegen.
        \item Stack sollte die Daten zuerst lesen und erst dann den SP inkrementieren.
    \end{enumerate}
\end{minipage}
%
\begin{minipage}{0.5cm}
	\-\
\end{minipage}
%
\begin{minipage}{9cm}
	 \includegraphics[width=9cm]{images/allocatedStack}  
\end{minipage}

\begin{minipage}[t]{9cm}
	\subsection{Shift and Rotate Instructions}
	\includegraphics[width=9cm]{images/shiftandrotate}\\
	LSL: Signed, unsigned Multiplikation mit $2^n$\\
	LSR: Unsigned Division mit $2^n$
	
\end{minipage}
%
\begin{minipage}[t]{0.5cm}
	\-\
\end{minipage}
%
\begin{minipage}[t]{9cm}
	\subsection{Bit-Field Processing Instructions}
		Der Cortex-M3/M4 Porzesor verf"ugt "uber viele verschiedene \textit{Bit-Field Processing Operations}-Instruktionen, einzelne sind folgend aufgelistet:\\

		\includegraphics[width=9cm]{images/bit-field-processing}
\end{minipage}


\begin{minipage}[t]{9cm}
	\subsection{Compare and Test}
	Die \textit{\textbf{COMPARE}} und \textit{\textbf{TEST}} Instruktionen aktualisieren die Flags im \textit{APSR}, welche f"ur den \textit{Conditional Branch} oder \textit{Conditional Execution} ben"otigt werden.\\

	\includegraphics[width=9cm]{images/compare-and-test}

\end{minipage}
%
\begin{minipage}[t]{0.5cm}
	\-\
\end{minipage}
%
\begin{minipage}[t]{9cm}
	\subsection{Program Flow Control}
	\subsubsection{Unconditional Branches}
	\textit{Unconditional Branches} sind Spr"unge zu einem Label, welche wie folgt implementiert werden k"onnen.\\
	
	\includegraphics[width=9cm]{images/branches}
	
	\subsubsection{Function Calls}
	F"ur den Funktionsaufruf wird die Instruktion \textit{Branch and Link} verwendet.\\
	
	\includegraphics[width=9cm]{images/Function_call}
\end{minipage}


\begin{minipage}{9cm}
	\subsubsection{Conditional Branches}
	\textit{Conditional Branches} sind bedingte Spr"unge, bei welchen nur zur angegeben Adresse gesprungen wird, wenn die Flags die Bedingung erf"ullen. Die Flags (APSR) werden immer vor einem \textit{Conditional Branch} mit einer \textbf{\textit{CMP}}-Instruktion evaluiert.\\
	
	\includegraphics[width=9cm]{images/Conditional_Branch1}
	
	Ein Beispiel:\\
	
	\includegraphics[width=9cm]{images/Conditional_Branch_Bsp}	
\end{minipage}
%
\begin{minipage}{0.5cm}
	\-\
\end{minipage}
%
\begin{minipage}{9cm}
	\includegraphics[width=9cm]{images/flags_conditional}
\end{minipage}

	





\clearpage
\section{V10}
\subsection{C/C++ Strukturen Umsetzen}
\subsubsection{Entscheidungen}
In Assembler-Sprache ist eine Entscheidung praktisch immer in einem 2-Stufigen Ablauf umgesetzt.
\begin{itemize}
    \item Benötigte Flags ermitteln
    \item Zugehörige bedingte Sprünge ausführen
\end{itemize}

Dabei wird nach folgendem Ablauf gearbeitet:\\
\vspace{-0.5cm}
\begin{enumerate}
    \item Vergleich: Zwei Werte werden subtrahiert, dabei wird nur auf die Flags geschaut.
    \item Anhand der Flags werden dann die bedingten Sprünge ausgeführt
\end{enumerate}

\subsubsection{Analyse von Hochsprachencode und Assembly-Code}
F"ur die Bestimmung der richtigen \textit{Conditional Branch} Instruktion kann wie am folgendem Beispiel gezeigt wird vorgegangen werden:

\begin{center}
	\includegraphics[width=10cm]{images/Bsp_cpp-assembly}
\end{center}

\begin{enumerate}
	\item Bestimmung des Datenformates der Variablen: \textit{signed} oder \textit{unsigned}\\
			$\Rightarrow$ Beispiel: signed
	\item Gleichung f"ur den \textit{Conditional Branch} aufstellen: \textbf{CMP Rn, Rm} entspricht Flags f"ur $Rn-Rm$\\
			$\Rightarrow$ Beispiel: $G1 > G2 \Leftrightarrow  0 > G2-G1$
\end{enumerate}
    
\begin{minipage}{9cm}
	\textbf{if-Block}\\
	F"ur einen Sprung in den if-Block w"are nun der Operator \color{red} \circled{$>$} \color{black} entscheidend.\\
	$\Rightarrow$ Conditional Branch Instruction: \textbf{BLT}\\
	Im Beispiel folgt jedoch der if-Block nach dem else-Sprung.
\end{minipage}
%
\begin{minipage}{0.5cm}
	\-\
\end{minipage}
%
\begin{minipage}{9cm}
	\textbf{else-if-Block}\\
	Die Gleichung wird zus"atzlich negiert:\\
	$0 > G2-G1$ = $\overline{0 \leq G2-G1}$\\
	F"ur einen Sprung in den else-if-Block w"are nun der Operator \color{red} \circled{$\leq$} \color{black} entscheidend.\\
	$\Rightarrow$ Conditional Branch Instruction: \textbf{BGE}
\end{minipage}

\subsubsection{Beispiele}
\begin{minipage}[t]{9cm}
	\textbf{IT-Instruktion}\\
	
	\includegraphics[width = 9cm]{images/IT-Instruction}
\end{minipage}
%
\begin{minipage}[t]{0.5cm}
	\-\
\end{minipage}
%
\begin{minipage}[t]{9cm}
	Wenn eine if..then..else Struktur nur wenige Sequenzen von Anweisungen enth"alt und keine \textit{Branches} kann dies auch mit der \textit{IT-Instruktion} umgesetzt werden, wobei diese maximal vier nachfolgende Anweisungen zu kontrollieren vermag.\\
	Die erste Instruktion im \textit{IT}-Block ist aktiviert, wenn der Condition-Code im Operandenfeld  erf"ullt ist. Die weiteren Anweisungen im Block werden durch anh"angen von eins bis drei Buchstaben mnemonisch gesteuert: T $\Rightarrow$ then; E $\Rightarrow$ else \\
		Allen Anweisungen im \textit{IT}-Block muss der entsprechende Condition-Code angeh"angt werden. Es werden \textbf{immer} gleich viele Taktzyklen ausgef"uhrt, egal welcher Pfad gefahren wird.
\end{minipage}
\newpage

\begin{minipage}[t]{9cm}
	\textbf{FOR-Schleife}\\
	
	\includegraphics[width=9cm]{images/For-Loop}
\end{minipage}
%
\begin{minipage}[t]{0.5cm}
	\-\
\end{minipage}
%
\begin{minipage}[t]{9cm}
	Eine universelle Schleife (Iteration) besteht im Allgemeinen aus vier Teilen:
	\begin{itemize}
		\item Initialisierung \color{green} \tikz \draw (0,0) rectangle (0.6,0.3); \color{black}
		\item Test auf Abbruch oder Fortsetzung 
		\item Update der Loop-Control Variable
		\item Loop-Body (zu wiederholender Programm-Code)
	\end{itemize}
	
	Bei FOR-Schleifen, welche die Laufvariable gegen Null vergleichen, k"onnen so manche Instruktionen erspart werden. So k"onnen statt der \textit{CMP}-Instruktion die \colorbox{yellow}{\textit{MOVS-/SUBS}-Instruktionen} verwendet werden, welche gerade das \textit{Zero-Flag} evaluieren. Solange die Laufvariable nicht im Body verwendet wird, kann jede FOR-Schleife damit optimiert werden.
\end{minipage}

\newpage













\section{V10}
\section{V11}
\subsection{Subroutinen}
\includegraphics[11cm]{images/subroutinen} 

\subsection{Architecure Producer Call Standart (AAPCS)}
\subsubsection{Regeln}
\begin{itemize}
    \item Bei Funktionsaufrufen werden die Register \textbf{R0-R3} als \textbf{Parameter} an eine C-Funktion verwendet
    \item Werden mehr als vier Funktionsparameter bent"otigt, werden diese vom Caller auf den Stack gelegt und auch wiedervom Caller entfernt.
    \item Die Funktionen müssen die Inhalte der Register \textbf{R4-R11}(falls benutzt) während der Ausführung sichern, um sie am Ende wieder rekonstruieren
    \item Der \textbf{Rückgabewert} einer Subroutine (8-bit, 16-bit,32-bit) wird in den \textbf{Registern R0} übertragen. Handelt es sich um einen 64-bit Rückgabewert, so sind die unteren 32-bit im Register R0 und die oberen 32-bit im Register R1 übertragen
    \item Mit PUSH und POP wird immer eine \textbf{gerade Anzahl von Registern auf dem Stack} gelegt bzw. vom Stack eine \textbf{8-byte Alignment} auf dem Stack einzuhalten
\end{itemize}

\subsubsection{Beispiel}
	Die Parameter"ubergabe bei Funktionen soll mit dem nachfolgenden Beispiel erkl"art werden. \\
	Die Funktion besitzt einen unsigned 8-Bit R"uckgabewert und sechs signed 8-Bit Parameter. \\($\Rightarrow$ 2 Parameter werden "uber den Stack "ubergeben)\\

\begin{minipage}{5cm}
	\begin{enumerate}
		\item \textbf{Stackpointer SP (R13)} wird um 8 nach nach unten verschoben
		\item  Variablen \textit{e} und \textit{f} werden vom Caller auf den Stack gelegt.
		\item \textit{R0 - R3} werden Parameter zugewiesen
		\item Sprung zum Label \textit{foo$\_$4}
		\item \textit{R2, R3, R7, LR} werden auf den Stack gelegt (\textit{SP} wird um 16 nach unten verschoben)
		\item \textit{R7} wird f"ur Zugriff auf \textit{e} und \textit{f} als Framepointer gesetzt.
		\item \textit{R0 - R3} werden auf den Stack gelegt
	\end{enumerate}
\end{minipage}
%
\begin{minipage}{0.25cm}
	\-\
\end{minipage}
%
\begin{minipage}{13cm}
	\includegraphics[width=13cm]{images/parameteruebergabe_stack}
\end{minipage}
\newpage
\clearpage
\section{V12, Serielle Schnittstelle}
\subsection{Grundlagen}
\subsubsection{Serielle vs. Parallele Daten"ubertragung}
\begin{minipage}{12cm}
	Bei einem seriellen Kommu nikationskanal werden die einzelnen Datenbits sequenziell hintereinander "uber eine Signalleitung "ubertragen. Im Gegensatz dazu k"onnen bei einem parallel Kommunikationskanal mehrere Datenbits auf einmal transferiert werden. Sowohl bei der seriellen wie auch bei der parallelen "Ubertragung ist dabei immer eine bestimmte Bit-"Ubetragungszeit $t_{bit}$ einzuhalten.
	
	Bei seriellen Kommunikationskan"alen werden allgemein zwei unterschiedliche Metriken unterschieden, die \textit{Bit Rate} und die \textit{Baud Rate}.
	
	Die \textit{Bit Rate} sagt aus, wie viele \textit{bits-per-second} oder \textit{bts} "uber den Datenkanal transferiert werden k"onnen.
	
	Die \textit{Baud Rate} sagt aus, wie viele sogennante \textit{Symbols} sich in einer Sekunde "uber den Datenkanal transferieren lassen. Ist jedes Symbol durch ein Bit repr"asentiert spricht man von einer \textit{Dualen Codierung} (Bit/s = bps $\equiv$ Baud)
\end{minipage}
%
\begin{minipage}{0.5cm}
	\-\
\end{minipage}
%
\begin{minipage}{6cm}
	\includegraphics[width=6cm]{images/serielle-parallel_grundlagen}
\end{minipage}

\subsubsection{Parallel-Seriell Umsetzung}
Der Mikroprozessor verarbeitet die Daten intern in aller Regel parallel. F"ur die serielle Daten"ubertragung ist daher eine koordinierte Parallel-Serien-Wandlung in eine daf"ur vorgesehenen I/O-Port erforderlich.
	\begin{center}
		\includegraphics[width=12cm]{images/seriell_parallel}
	\end{center}
	
\subsubsection{Aufbau von Datenkan"alen}Kommunikatiosnkan"ale k"onnen auf verschiedene Arten implementiert werden. So k"onnen drahtgebundene Kan"ale beispielsweise als \textit{Single-Ended} oder als \textit{Differential Link} umgesetzt werden.\\
	
	\begin{minipage}[t]{9cm}
		\textbf{Single Ended (a)}\\
		Bei einer Single-Ended Verbindung werden einzelne, getrennte Signalleitungen ben"otigt, sowie eine Referenzleitung mit Ground-Potenzial
	\end{minipage}
	%
	\begin{minipage}[t]{0.5cm}
		\-\
	\end{minipage}
	%
	\begin{minipage}[t]{9cm}
		\textbf{Differential Link (b)}\\
		Bei einer Differential Verbindung wird der Datenlink durch eine Differenzspannung zwischen zwei zusammengeh"orenden Signalleitungen V(+) und V(-) repr"asentiert. Dies ist deutlich robuster als Single-Ended. F"ur eine serielle Differential Datenschnittstelle gen"ugen insgesamt drei Leitungen.
	\end{minipage}
	
	\begin{center}
		\includegraphics[width = 16cm]{images/diff_link}
	\end{center}

\newpage
\subsubsection{Simplex, Half- und Full-Duplex}
Datenkan"ale werden mit \textit{Simplex, Half-Duplex} oder \textit{Full-Doplex} bezeichnet, je nach Art der Konektivit"at.

		\textbf{Simplex}\\
		Ein Simplex serieller Kanal "ubertr"agt permanent in nur einer Richtung, "uber eine dedizierte Verbindung. An einem Ende des Kommunikationskanals arbeitet ein
			Transmitter, w"ahrend auf der anderen Seite ein Receiver steht. Simplex Kan"ale haben keine M"oglichkeit, den Empfang der Daten zu quittieren.
		\begin{center}
			\includegraphics[width=12cm]{images/Simplex_serial}
		\end{center}
		

		\textbf{Half-Duplex}\\
		Ein Half-Duplex serieller Kanal verf"ugt ebenfalls nur "uber eine einzige Verbindung. Diese erlaubt jedoch eine beidseitige Kommunikation, aber nur in einer Richtung zur gleichen Zeit. Auf beiden Seiten des Kommunikationskanals steht ein serieller Transceiver, der sowohl als Sender wie auch als Empf"anger arbeiten kann. Wird die "Ubertragungsrichtung ge"andert, so mu"ussen die beiden Transceiver ihre Betriebsart wechseln. Das Umschalten der Datenrichtung erfordert klare Regeln auf beiden Seiten, um beispielsweise ein gleichzeitiges Senden zu vermeiden.
		\begin{center}
			\includegraphics[width=12cm]{images/half_duplex}
		\end{center}
		
	 
	 	\textbf{Full-Duplex}\\
	 	Full-Duplex serielle Kan"ale verf"ugen "uber zwei separate Datenlinks; einer um Daten zu senden und ein anderer um Daten zu empfangen. Damit ist eine gleichzeitige Kommunikation in beide Richtungen m"oglich. Auf beiden Seiten des Kommunikationskanals steht ein serieller Transceiver, der zeitgleich als Sender und Empf"anger arbeitet.
	 	\begin{center}
			\includegraphics[width=12cm]{images/full_duplex}
		\end{center}

\newpage
\subsection{Synchrone Daten"ubertragung}
Synchrone serielle Kan"ale werden dadurch gekennzeichnet, dass Sender und Empf"anger auf das gleiche Clock-Signal synchronisiert sind $\Rightarrow$ zus"atzliche Clock-Leitung. Synchrone Kan"ale arbeiten "ublicherweise in einer \textit{Master/Slave-Beziehung}.\\
	\begin{center}
		\includegraphics[width=12cm]{images/synch_full_duplex}
	\end{center}
	
\subsubsection{Daten"ubertragung}
Beim der synchronen seriellen Datenkanal werden die Daten typischerweise in Bl"ocken mit variabler L"ange "ubertragen. Parallel dazu erfolgt die "Ubertragung der Clock Information. Dabei ist wichtig, dass Sender und Empf"anger auf die gleiche Flanke Daten "ubernehmen.

Im folgenden Bild ist exemplarisch eine ganze Message mit drei Data Packets (Datagrams) dargestellt: Das Synchronisationszeichen (Sync) kennzeichnet als Header den Beginn eines zusammengeh"orenden Data Pakets. Grunds"atzlich wird nach jedem Datenblock (Body) das Zeichen ETB (End of Transmission Block) als Footer "ubertragen. Das Ende der gesamten Message wird durch ein EOT (End of Transmission) gekennzeichnet. Das ETB kann vor einem EOT je nach Protokoll entfallen.
	\begin{center}
		\includegraphics[width=12cm]{images/synch-serielle-data-bsp}
	\end{center}

\subsection{Asynchrone Daten"ubertragung}
Beim asynchronen seriellen Datenkanal laufen auf der Sender- und Empf"angerseite zwei unabh"angige Clock-Generatoren. Dabei m"ussen diese auf die selbe Baud-Rate eingestellt werden. 
	\begin{center}
		\includegraphics[width=12cm]{images/asynch_serial.png}
	\end{center}

Auf die fallende Datenflanke des Datenframes wird der jeweilige Clock-Generator synchronisiert. Dabei haben Datenpakete folgenden Aufbau:\\
	\begin{minipage}{11cm}
		\includegraphics[width=11cm]{images/datenframe-uart}\\
		Datenframe einer UART Schnittstelle (7-Bit, Odd Parity)
	\end{minipage}
	%
	\begin{minipage}{0.5cm}
		\-\
	\end{minipage}
	%
	\begin{minipage}{7cm}
		Die beteiligten Kommunikationsschnittstellen ben"otigen folgende Informationen um ihre Interfaces auf den asynchronen seriellen Bitstrom abzustimmen:
		\begin{itemize}
			\item Baudrate
			\item Anzahl Daten-Bits (5 bis 8)
			\item Parity Information (Even / Odd / none) Das Paritybit erg"anzt immer auf \textit{Odd} oder \textit{Even}.
			\item Anzahl Stop-Bit (1 / 1.5 / 2)
		\end{itemize}
	\end{minipage}\\

Aus den Informationen der Kommunikationsschnittstelle l"asst sich zudem die Nutzdatenrate bestimmen.
	\begin{equation*}
		Nutzdatenrate = \frac{\#Databit \cdot Baudrate}{\#Startbit + \#Databit + \#Paritybit + \#Stopbit} \cdot \frac{1 Byte}{8 Bit}
	\end{equation*}
	
\newpage
\subsubsection{Datenflusssteuerung}
Bei der seriellen Kommunikation muss oft der Datenfluss gesteuert werden. Dies ist zum Beispiel dann dringend notwendig, wenn empf"angerseitig der Datenpuffer voll wird und ein Daten"uberlauf droht. Dieser Zustand muss an den Datensender signalisiert werden. $\Rightarrow$ HW- und SW-Handshake\\

	\begin{minipage}[t]{9cm}
		\textbf{Hardware Handshaking}\\
		Mit Hardware-Signalen teilen sich die Kommunikationspartner gegenseitig mit, ob sie bereit sind weitere Daten aufzunehmen oder nicht.
		\begin{center}
			\includegraphics[width=5cm]{images/HW-handshake}\\
			\textit{RTS (request to send), CTS (clear to send)}
		\end{center}
	\end{minipage}	
	%
	\begin{minipage}[t]{0.5cm}
		\-\
	\end{minipage}
	%
	\begin{minipage}[t]{9cm}
		\textbf{Software Handshaking}\\
		Beim Software Handshaking kommen anstelle von zus"atzlichen Handshake-Leitungen zwei extra f"ur diesen Zweck reservierte ASCII-Zeichen zur Anwendung:
		\begin{itemize}
			\item \textbf{XON} ASCII DC1 0x11
			\item \textbf{XOFF} ASCII DC3 0x13
		\end{itemize}
		\begin{center}
			\includegraphics[width=5cm]{images/SW-handshake}
		\end{center}
	\end{minipage}
\subsection{Standardisierte serielle Interfaces}
\includegraphics[width=7cm]{images/uebertragungsraten_seriellen_schnittstellen}
\includegraphics[width=11cm]{images/uebersicht-serielle-schnittstellen}
\section{V13}
\subsection{Kommunikation zwischen Rechnersystemen}
F"ur eine rechner"ubergreifende Kommunikation stehen heutzutage verschiedene Protokolle und Spezifikationen zur Verf"ugung. Trotz der Vielfalt basieren heute die meisten Implementationen auf den Grundkonzepten des OSI-Referenzmodells und legen damit schon die entscheidenden Weichen f"ur geordnete Kommunikationsabl"aufe und saubere Netzwerkstrukturen.

\begin{minipage}{9cm}
\subsubsection{ISO/OSI-Modell}
	Das OSI-Referenzmodell beschreibt modellhaft die Kommunikation in \textbf{7}
 hierarchischen Schichten (Layer) unterschiedlicher Abstraktion. Hierbei stellt eine Schichht \textbf{i} Dienstleistungen f"ur die dar"Uber liegende Schicht \textbf{i+1} zur Verf"ugung, sie selbst bezieht Dienstleistungen von der unter ihr liegende Schicht \textbf{i-1}.
\end{minipage}
%
\begin{minipage}{0.5cm}
	\-\
\end{minipage}
%
\begin{minipage}{9cm}
	\includegraphics[width=8cm]{images/OSI-Referenzmodell}
\end{minipage}

 Das OSI-Referenzmodell stellt eine Konstruktionsanleitung f"ur Netzwerksoftware da. "Uber insgesamt sieben Layer sind abstrakte Funktionen definiert, die aufeinander aufbauen. Beim Datentransport werden die Nutzdaten in den jeweiligen Protokollrahmen eingebettet und der unteren Schicht weiter gereicht. $\Rightarrow$ Kapselung\\

 \begin{minipage}{10.5cm}
 	\includegraphics[width=10 cm]{images/OSI-Layer}
 \end{minipage}
 %
 \begin{minipage}{0.5cm}
 	\-\
 \end{minipage}
 %
 \begin{minipage}{7.5cm}
 	\textbf{Kapselung}\\
 	\includegraphics[width=7 cm]{images/OSI-Kapselung}
 \end{minipage}
 
 
 \begin{minipage}[t]{9cm}
 	\textbf{Bit"ubertragung (Physical Layer)}\\
	- "Ubertragung der abstrakten Informationseinheiten '0' und '1' mittels physikalischer Mittel "uber ein Medium.\\
    - "Ubetr"agt einen Bitstrom von Punkt A nach B, ohne dessen Informationsgehalt zu modifizieren. Diese "Ubertragung erfolgt ungesichert, also ohne Fehlertoleranz.
 \end{minipage}
 %
 \begin{minipage}[t]{0.5cm}
 	\-\
 \end{minipage}
 %
 \begin{minipage}[t]{9cm}
 	\textbf{Sicherung (Data Link Layer)}\\
	- Unver"anderte "ubertragen von Daten, d.h. ohne Fehler und in der richtigen Reihenfolge. Sie stellt den "ubergeordneten Schichten gesicherte Verbindungen zur Verf"ugung.\\
	- Durchf"uhrung der Flussregelung, also z.B. die Verhinderung eines Daten"uberlaufs beim Empf"anger (z.B. XON/XOFF-Protokoll).
 \end{minipage}
 
 
 \begin{minipage}[t]{9cm}
 	\textbf{Vermittlung (Network Layer)}\\
	- Verbindet Endsysteme miteinander (z.B. Computer P und Q). Dies kann die Benutzung eines oder mehrerer Kommunikationsnetze einschliessen.\\
	- W"ahlt geeigneten Route (Leitweglenkung, Routing) f"ur das Netzwerk, da meistens verschiedene vorhanden.
\end{minipage}
 %
 \begin{minipage}[t]{0.5cm}
 	\-\
 \end{minipage}
 %
 \begin{minipage}[t]{9cm}
 	\textbf{Transport (Transport Layer)}\\
- Stellt den Anwendungen transparente Datenkan"ale zur Verf"ugung, sie schafft damit Verbindungen zwischen Anwendungsprozessen auf verschiedenen Systemen. Die Transparenz besteht darin, dass von der "ubergeordneten Schicht "ubernommene Daten (Bitstrom) unver"andert "ubertragen werden.
 \end{minipage}

\begin{minipage}[t]{9cm}
 	\textbf{Sitzung (Session Layer)}\\
	- Stellt Mittel f"ur eine geordnete Kommunikationsbeziehung zur Verf"ugung. Dazu geh"ort die Er"offnung, Durchf"uhrung und Beendigung der Session. Jede Session kennt die Phasen Verbindungsaufbau, Datentransfer und Verbindungsabbau.
 \end{minipage}
 %
 \begin{minipage}[t]{0.5cm}
 	\-\
 \end{minipage}
 %
 \begin{minipage}[t]{9cm}
 	\textbf{Darstellung (Presentation Layer)}\\
	- Wird ben"otigt, wenn es um die Beschreibung von Daten geht, sofern diese Beschreibung nicht schon Teil der Applikation selbst ist. Zu diesem Zweck wird die Datenbeschreibung ASN1 (Abstract Syntax Notation One) eingesetzt.
 \end{minipage}
 
 
 \textbf{Anwendung (Application Layer)}\\
"Uber diese Schicht werden den Anwendungen die Kommunikationsdienste in anwendungsunabh"angiger Form zug"anglich gemacht.

\subsection{Allgemeiner Ablauf von Exceptions und Interrupts}
\begin{minipage}{8cm}
Interrupts werden in der Regel von der umgebenen Peripherie oder externen Input-Pins generiert und als Ereignis der CPU-Infrastruktur signalisiert, welche dann eine Handler-Routine einschalten. \\
	\textbf{Siehe} \nameref{Exceptions} Seite: \pageref{Exceptions}
\end{minipage}
%
\begin{minipage}{0.5cm}
	\-\
\end{minipage}
%
\begin{minipage}{10cm}
\includegraphics[width=10cm]{images/interruptablauf} 
\end{minipage}

\begin{minipage}{8cm}
	Der allgemeine Ablauf gliedert sich in mehrere Teilschritte:\\
	\begin{enumerate}
		\item Die Peripherie meldet einen Interrupt Request (IRQ) beim Prozessor an
		\item Prozessor beendet die laufende Instruktion (Background)
		\item Prozessor f"uhrt einen Exception Handler oder Interrupt Service Routine (ISR) aus (Foreground)
		\item Prozessor f"ahrt bei der n"achsten Instruktion im Background weiter
	\end{enumerate}
\end{minipage}
%
\begin{minipage}{0.5cm}
	\-\
\end{minipage}
%
\begin{minipage}{10cm}
	\includegraphics[width=10cm]{images/HW-Interrupt}
\end{minipage}

\subsection{Alternative Ans"atze}
\begin{center}
	\includegraphics[width=16cm]{images/alternative-lesezugriffe}
\end{center}

\subsubsection{Blind-Cycle}
In der Software wird immer eine fixe Zeit abgewartet, bis die Daten an der
	Hardware abgeholt werden. Diese fixe Blind-Cycle Time ist so zu bemessen, dass die Daten von der HW in jedem Falle bereitgestellt werden k"onnen (worst case scenario).

\subsubsection{Busy-Wait}
In der Software wird solange in einem Loop verweilt, bis die Hardware "uber
eine Status-Information signalisiert, dass die Daten bereitliegen. Damit k"onnen die empfangenen Daten mit Sicherheit abgeholt werden. Es etabliert sich so ein dynamisches Handshaking zwischen der Software und der Hardware.


\subsubsection{Interrupt}
Die Hardware meldet "uber einen IRQ asynchron zum aktuellen
Programmablauf, dass die Daten bereitliegen. Die zugeh"orige ISR unterbricht den
sequenziellen Ablauf im Background-Programm. In der ISR (Foreground) werden die
Daten von der Hardware abgeholt und in einen FIFO-Buffer gelegt. Die Daten k"onnen danach im Background asynchron aus dem FIFO-Buffer gelesen und verarbeitet werden.
\clearpage
\section{V14}
\subsection{Spezielle Eigenschaften des NVIC}\label{NVIC}
\begin{minipage}{9cm}
	Alle Cortex-M Prozessoren enthalten einen \textit{Nested Vectored Interrupt Controller (NVIC)} f"ur das Interrupt-Handling. Neben den klassischen HW-initiierten Interrupts (IRQ) gibt es eine ganze Anzahl von Exceptions, die vom NVIC ebenfalls gehandhabt werden. Dazu geh"oren z.B. Fault Exceptions, NMI Software-Interrupts (SVC), SysTick Timer, usw.
\end{minipage}
%
\begin{minipage}{0.5cm}
	\-\
\end{minipage}
%
\begin{minipage}{9cm}
	\includegraphics[width=9cm]{images/nvic-cortex-m3} 
\end{minipage}

\subsection{Cortex-M3 Exceptions und Priority-Levels}
\begin{minipage}{9cm}
	Der erste Eintrag in der Vektor-Tabelle (Address-Offset: 0x00) enth"alt immer den Initialwert f"ur den Main Stack Pointer (MSP). Darauf folgt die Einsprung-Adresse f"ur den Reset-Handler. Durchl"auft der Cortex-M seine Reset-Sequenz, so werden diese beiden Eintr"age in den MSP bzw. in den PC geladen. Die n"achsten Eintr"age in der Vektor-Tabelle sind f"ur den Nonmaskable Interrupt (NMI), sowie vier verschiedene Faults reserviert. Weiter oben in der Vektor-Tabelle folgen die
		Adresseintr"age f"ur die weiteren System Exception-Handler.
\end{minipage}
%
\begin{minipage}{0.5cm}
	\-\
\end{minipage}
%
\begin{minipage}{9cm}
	\includegraphics[width=9cm]{images/NVICExcp1} 
\end{minipage}

\vspace{10pt}
\begin{minipage}[t]{9cm}
	\textbf{Usage Fault}\\
	Ein Usage Fault tritt auf, wenn der auszuf"uhrende Application-Code im Cortex-M3 zu einem un"uberwindbaren Fehler f"uhrt. Eine typische Ursache ist beispielsweise, wenn der Prozessor einen ung"ultigen OpCode auszuf"uhren versucht. Weitere Ursachen f"ur einen Usage Fault k"onnte eine Division durch Null sein.\\
	
	\textbf{Bus Fault}\\
	Ein Bus Fault wird ausgel"ost, wenn ein Fehler in der AHB-Bus Matrix erkannt wird. M"ogliche Gr"unde f"ur einen Bus Fault k"onnte eine falscher Memory-Bereich oder eine falsche Gr"osse f"ur einen Datentransfer sein.
\end{minipage}
%
\begin{minipage}[t]{0.5cm}
	\-\
\end{minipage}
%
\begin{minipage}[t]{9cm}
	\textbf{Memory Manager Fault}\\
	Die Memory Protection Unit (MPU) kann in den folgenden F"allen einen Memory
	Manager Fault ausl"osen:\\
	- Accessing an MPU region with the wrong privilege level\\
	- Writing to a read-only region\\
	
	\textbf{Hard Fault}\\
	Ein Hard Fault kann auf zwei Arten auftreten: (1) Wenn ein Bus Fault tritt auf, w"ahrend dem die Vektor-Tabelle gelesen wird. (2) Der Hard Fault wird durch die Eskalationen eines anderen, nicht behandelten Faults aktiviert.
\end{minipage}

\newpage
\subsubsection{Exception-Priority}
\begin{minipage}{9cm}
	Bei den Cortex-M Prozessoren kann individuell festgelegt werden, ob eine Exception zugelassen oder nicht zugelassen werden soll. Dazu wird der aktuelle Priority Level mit der konfigurierten Priority der Exception gegeneinander verglichen. Eine h"oher priorisierte Exception (kleinerer Priority Level) kann eine Exception mit tieferer Priorit"at (gr"osserer Priority Level) unterbrechen. Dieses \textit{preemptive} Verhalten
	wird allgemein als \textbf{Nested Exception/Interrupt Szenario} bezeichnet.\\

Das Priority-Level kann abh"angig vom Modell bis zu 8 Bit gross sein, und kann mit der \textit{Group-Priority} in die \textit{Preempt-Priority} und die \textit{Sub-Priority} aufgeteilt werden. $\Rightarrow$ Festlegung der Anzahl Priority-Level pro Gruppe
\end{minipage}
%
\begin{minipage}{0.5cm}
	\-\
\end{minipage}
%
\begin{minipage}{9cm}
	\includegraphics[width=9cm]{images/group-priority}
\end{minipage}


\vspace{10pt}
\begin{minipage}[t]{9cm}
	\textbf{Preempt-Priority}\\
	Legt fest ob ein Interrupt erfolgen kann, wenn bereits ein anderer Interrupt-Handler am laufen ist.
\end{minipage}
%
\begin{minipage}[t]{0.5cm}
	\-\
\end{minipage}
%
\begin{minipage}[t]{9cm}
	\textbf{Sub-Priority}\\
	Wird verwendet, wenn zwei Exceptions der selben Preempt Priority gleichzeitig auftreten. In solchen F"allen wird zuerst die Exception mit der h"oheren Sub-Priority (tieferer numerischer Wert) bearbeitet.
\end{minipage}

\subsubsection{Interrupt-Pending and Activation}
\begin{center}
	\includegraphics[width=15cm]{images/interrupt-pending}
\end{center}

\subsubsection{Tail Chaining}
Wenn eine Exception auftritt während bereits eine anderen Exception-Behandlung mit gleicher oder höherer Priorität läuft, so wird die neue Exception hinten angestellt. Nach Abschluss des laufenden Exception Handlers, kann die CPU sofort den neuen Exception Request behandeln

\subsubsection{Late arrival}
Wenn der Prozessor einen auftretenden Exceptionrequest akzeptiert, dann startet er die Stacking-Sequenz. Kommt während dem Stacking eine weitere Exception mit höherer Priorität hinzu, so kann diese Late-Arrival-Exception noch bevorzugt behandelt werden.

\subsubsection{POP Preemption} 
Diese Funktion stellt gewissermassen eine Umkehrung des Late-Arrivals dar. Wenn eine Exception Request während dem Unstacking auftritt, so wird das Unstacking abgebrochen, und sofort VectorFetch und Instruction Fetch für den neuen Request durchgeführt. $\rightarrow$ Geschwindigkeitsoptimierung\\


\clearpage
\section*{Anhang}
\subsection*{Glossar,Abkürzung}
    \includegraphics[width=0.75\linewidth]{images/glossar}  

\end{document}
