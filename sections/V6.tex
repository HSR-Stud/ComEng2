\section{V6}
\begin{multicols}{2}
    \begin{minipage}{\linewidth}
        \subsection{Exceptions and Interrupts}\label{Exceptions}
        Der NVIC verarbeitet bis 240 IRQ und einen NMI\\
        \begin{tabular}{ll}
            normaler Programmablauf& \rightarrow Background  \\ 
            Exception-Handler& \rightarrow Foreground  \\ 
        \end{tabular} 
    \end{minipage}
    
    \includegraphics[width=\linewidth]{images/NVICExcp}
\end{multicols}
Exception-Handler für Interrupts werden als Interrupt Service Routine (ISR) bezeichnet.

\subsection{Reset und Reset-Sequenzen}
\subsubsection{Reset}
Es gitb 3 Arten von Reset:\\
\begin{tabular}{ll}
    \textbf{Power-on Reset}  & Resettet den gesammten \mu C, auch alle Preripherien und Debug-Komponenten \\ 
    \textbf{System Reset}    & Resettet nur den Prozessor und die Periferien, aber nicht die Debug-Komponenten \\ 
    \textbf{Processer Reset}& Resettet nur den Prozessor\\
\end{tabular}

\subsubsection{Reset Sequenz}
     \includegraphics{images/resetsequenz}
\subsection{Spezial-Register}     
Register um Exceptions ein oder auszuschalten:\\
\textbf{\rightarrow PRIMASK,FAULTMASK,BASEPRI}
\begin{multicols}{2}
    \subsubsection{PRIMASK}
    \begin{itemize}
        \item 1-bit Register
        \item Wenn das aktiv ist, werden NMI-Interrupts erlaubt
        \subitem \rightarrow alle anderen Interrupts werden überdeckt
        \subitem \rightarrow Default-Wert= 0, also deaktiviert
    \end{itemize}
    
    \subsubsection{FAULTMASK}
    \begin{itemize}
        \item 1-bit Register
        \item Wenn das aktiv ist, werden nur noch NMI-Interrups akzeptiert.\newline
        Alle anderen Interrupts und Exceptioln-Handlings werden deaktiviert
        \subitem \rightarrow Default-Wert = 0
    \end{itemize}
\end{multicols}

\subsubsection{BASEPRI}
\begin{itemize}
    \item Register das bis zu 8 Bits enthalten kann
    \item definiert eine Prioritätsstufe
    \item Hohe Stufe = Hohe Priorität
    \item Wenn das gesetzt wird, werden alle Interrupts mit gleicher oder tieferer Stufe deaktiviert
\end{itemize}

\subsubsection{Control-Register}
Das Kontroll-Register definiert:
\begin{enumerate}
    \item Die Auswahl zwischen MSP (Main-SP) und PSP (Process-SP)
    \item Die Zugriffsstufe und Thread-Mode
    \subitem (Ob Privilegd oder unprivilegd)
\end{enumerate}
\begin{multicols}{2}
    \includegraphics[width=\linewidth]{images/StackPointerAuswahl}
    
    \includegraphics[width=\linewidth]{images/controlRegister}
\end{multicols}