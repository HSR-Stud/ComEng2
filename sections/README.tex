\thispagestyle{empty}
\setcounter{page}{0} %Set PageNumber to 0
{\huge README }
\section*{Beschreibung}
Zusammenfassung für Computer Engineering 2 auf Grundlage der Vorlesung FS 16 von Erwin Brändle \newline
Bei Korrekturen oder Ergänzungen wendet euch an einen der Mitwirkenden.

\section*{Modulschlussprüfung}
Kompletter Stoff aus Skript, Vorlesung, Übungen und Praktikum
{\scriptsize 
    \begin{itemize}
        \item Vorlesungsskript CompEng2 V1.2 komplett
        \subitem(die Kapitel 2 und 7 sind im Selbststudium individuell aufzuarbeiten)
        \item Korrigenda zum Skript, falls eine solche vorliegt
        \item Übungen im Vorlesungsskript
        \item Inhalt aller Praktika (inkl. Pre-/Post-Lab Übungen)
        \item in Vorlesungen und Praktika zusätzlich vermittelte Informationen
        \item Inhalt und Umgang mit dem Quick-Reference/Summary V1.2
    \end{itemize}
}
\textbf{Die Prüfung besteht aus 2 Teilen:}\newline
% \usepackage{array} is required
\begin{tabular}{p{1.5cm} p{3cm} p{10cm}}
    \textbf{ 1.Teil}   & closed Book & Theoretische Fragen zum ganzen Prüfungsinhalt \\ 
    \textbf{ 2.Teil}   & semi-open book & Aufgaben im Stil der Übungen, Praktika und der in den Vorlesungen gelösten Aufgaben \\ 
\end{tabular} 

\subsection*{Plan und Lerninhalte}
Fokus: ARM Cortex-M Architektur
{\scriptsize 
    \begin{itemize}
        \item RISC-Architektur, Core-Components, Register Model, Memory Model, Exception Model, Instruction Set Architecture
        \item Konzept und Umsetzung der vektorisierten Interrupt Verarbeitung
        \item Abbildung von typischen C Programmstrukturen und Speicherklassen in das Programmiermodell der CPU
        \item Systembus: Address-, Daten-, Control-Bus, Adressdekodierung, Memory- und I/O-Mapping
        \item Speicher- und ausgesuchte Peripherieschnittstellen
    \end{itemize}
}
\vfill
\section*{Contributors}
\begin{tabular}{ll}
    Luca Mazzoleni& luca.mazzoleni@hsr.ch \\ 
    Stefan Reinli & stefan.reinli@hsr.ch \\ 
\end{tabular} 

{\scriptsize 
    \section*{License}
    \textbf{Creative Commons BY-NC-SA 3.0}
    
    Sie dürfen:
    \begin{itemize}
        \item Das Werk bzw. den Inhalt vervielfältigen, verbreiten und öffentlich
        zugänglich machen.
        \item Abwandlungen und Bearbeitungen des Werkes bzw. Inhaltes anfertigen.
    \end{itemize}
    Zu den folgenden Bedingungen:
    \begin{itemize}
        \item Namensnennung: Sie müssen den Namen des Autors/Rechteinhabers in der von ihm
        festgelegten Weise nennen.
        \item Keine kommerzielle Nutzung: Dieses Werk bzw. dieser Inhalt darf nicht für
        kommerzielle Zwecke verwendet werden.
        \item  Weitergabe unter gleichen Bedingungen: Wenn Sie das lizenzierte Werk bzw. den
        lizenzierten Inhalt bearbeiten oder in anderer Weise erkennbar als Grundlage
        für eigenes Schaffen verwenden, dürfen Sie die daraufhin neu entstandenen
        Werke bzw. Inhalte nur unter Verwendung von Lizenzbedingungen weitergeben,
        die mit denen dieses Lizenzvertrages identisch oder vergleichbar sind.
    \end{itemize}
    Weitere Details: http://creativecommons.org/licenses/by-nc-sa/3.0/ch/
}
%If we meet some day, 
%and you think this stuff is worth it, you can buy me a beer in return.
\clearpage
\pagenumbering{arabic}% Arabic page numbers (and reset to 1)