\section{V3}
\subsection{Halbleiter Speicher}
\begin{multicols}{2}
\textbf{Zentraler Speicher}
\begin{itemize}
    \item direkt am Bussystem angeschlossen
\end{itemize}
\textbf{Peripherer Speicher}
\begin{itemize}
    \item über I/O-Schnittstelle angeschlossen
\end{itemize}

\includegraphics[width=10cm]{images/halbleiterfam}
\end{multicols}

\begin{multicols}{2}
\subsubsection{ROM-Festwertspeicher}
\includegraphics[width=8cm]{images/ROM}

\subsubsection{RAM-Speicher-/Lese-Speicher}
\includegraphics[width=8cm]{images/RAM}
\end{multicols}

\subsection{Speicherorganisation}
\begin{multicols}{2}
\includegraphics[width=8cm]{images/LittleBigEndian}

\subsubsection{I/O - Schnittstelle}
\includegraphics[width=8cm]{images/IOSchnittstelle}
\end{multicols}

\includegraphics{images/Speicherraumadressierung}
\section{Cortex}
\subsection{Cortex M Varianten}
\begin{multicols}{2}
\textbf{Cortex M0 und M0+}
    \begin{itemize}
        \item kleinster Vertreter der CortexFam
        \item Ersatz von 8Bit- uC
        \end{itemize}                     
 \textbf{Cortex M1}
    \begin{itemize}
        \item als Softcore Implementiert
        \item Vergleichbar mit Cortex-M0
    \end{itemize}
\end{multicols}
\begin{multicols}{2}
   \textbf{Cortex M3}     
     \begin{itemize}
         \item erster Vertreter der CortexFam
         \item 32 Bit Architektur
         \item ersetzt 8 \& 16 Bit uC
         \item Thumb ISA (Instruction Set Architecure)\newline
         Mix aus 16 und 32BIt langen anweisungen
     \end{itemize}   
               
   \textbf{Cortex M4} 
    \begin{itemize}
        \item vergleichbar mit M3 jedoch mit
        \qquad\item Digital Signal Processing (DSP)
        \qquad\item Floating Point Unit (FPU)
        \newline
      \end{itemize}  
\end{multicols}
\clearpage
\begin{multicols}{2}
\includegraphics[width=\linewidth]{images/cortexmfam}

\includegraphics[width=\linewidth]{images/cortexmcomp}
\end{multicols}

\begin{multicols}{3}
    \textbf{Cortex-A}
    \begin{itemize}
        \item HighEnd Anwendungen und Betriebssysteme
        \item hohe Rechenleistung
        \item Chache Memory
    \end{itemize}
    
    \textbf{Cortex-R}
    \begin{itemize}
       \item Echtzeitfähigkeit
       \item hohe Zuverlässigkeit
       \item System on Chip (SOC) 
    \end{itemize}  
    
        \textbf{Cortex-M}
        \begin{itemize}
            \item Speziell für \mu C-Markt
            \item Low Cost, Low Energy
            \item System on Chip (SOC) 
          \end{itemize}             
\end{multicols}
\subsubsection{Vorteile der Cortex-M-Prozessoren}
\begin{itemize}
    \item Low Power
        \subitem < 200\mu A / MHz
    \item Performance
        \subitem >1.25 DMIPS / MHz
    \item Energy Efficiency
        \subitem low Power, high performance
    \item Code Density
        \subitem Thumb 2 Befehlssatz
    \item Interrupts
        \subitem 240 Interrupts
    \item Easy of Use, C Friedly
    \item Scalability
    \item Debug Features
    \item Software portability and Reusebility
    \item OS Support
    \item Choices (Derivers, Tools, OS,..)    
\end{itemize}






















